% Options for packages loaded elsewhere
\PassOptionsToPackage{unicode}{hyperref}
\PassOptionsToPackage{hyphens}{url}
\PassOptionsToPackage{dvipsnames,svgnames,x11names}{xcolor}
%
\documentclass[
  letterpaper,
  DIV=11,
  numbers=noendperiod]{scrartcl}

\usepackage{amsmath,amssymb}
\usepackage{iftex}
\ifPDFTeX
  \usepackage[T1]{fontenc}
  \usepackage[utf8]{inputenc}
  \usepackage{textcomp} % provide euro and other symbols
\else % if luatex or xetex
  \usepackage{unicode-math}
  \defaultfontfeatures{Scale=MatchLowercase}
  \defaultfontfeatures[\rmfamily]{Ligatures=TeX,Scale=1}
\fi
\usepackage{lmodern}
\ifPDFTeX\else  
    % xetex/luatex font selection
\fi
% Use upquote if available, for straight quotes in verbatim environments
\IfFileExists{upquote.sty}{\usepackage{upquote}}{}
\IfFileExists{microtype.sty}{% use microtype if available
  \usepackage[]{microtype}
  \UseMicrotypeSet[protrusion]{basicmath} % disable protrusion for tt fonts
}{}
\makeatletter
\@ifundefined{KOMAClassName}{% if non-KOMA class
  \IfFileExists{parskip.sty}{%
    \usepackage{parskip}
  }{% else
    \setlength{\parindent}{0pt}
    \setlength{\parskip}{6pt plus 2pt minus 1pt}}
}{% if KOMA class
  \KOMAoptions{parskip=half}}
\makeatother
\usepackage{xcolor}
\setlength{\emergencystretch}{3em} % prevent overfull lines
\setcounter{secnumdepth}{5}
% Make \paragraph and \subparagraph free-standing
\ifx\paragraph\undefined\else
  \let\oldparagraph\paragraph
  \renewcommand{\paragraph}[1]{\oldparagraph{#1}\mbox{}}
\fi
\ifx\subparagraph\undefined\else
  \let\oldsubparagraph\subparagraph
  \renewcommand{\subparagraph}[1]{\oldsubparagraph{#1}\mbox{}}
\fi


\providecommand{\tightlist}{%
  \setlength{\itemsep}{0pt}\setlength{\parskip}{0pt}}\usepackage{longtable,booktabs,array}
\usepackage{calc} % for calculating minipage widths
% Correct order of tables after \paragraph or \subparagraph
\usepackage{etoolbox}
\makeatletter
\patchcmd\longtable{\par}{\if@noskipsec\mbox{}\fi\par}{}{}
\makeatother
% Allow footnotes in longtable head/foot
\IfFileExists{footnotehyper.sty}{\usepackage{footnotehyper}}{\usepackage{footnote}}
\makesavenoteenv{longtable}
\usepackage{graphicx}
\makeatletter
\def\maxwidth{\ifdim\Gin@nat@width>\linewidth\linewidth\else\Gin@nat@width\fi}
\def\maxheight{\ifdim\Gin@nat@height>\textheight\textheight\else\Gin@nat@height\fi}
\makeatother
% Scale images if necessary, so that they will not overflow the page
% margins by default, and it is still possible to overwrite the defaults
% using explicit options in \includegraphics[width, height, ...]{}
\setkeys{Gin}{width=\maxwidth,height=\maxheight,keepaspectratio}
% Set default figure placement to htbp
\makeatletter
\def\fps@figure{htbp}
\makeatother
% definitions for citeproc citations
\NewDocumentCommand\citeproctext{}{}
\NewDocumentCommand\citeproc{mm}{%
  \begingroup\def\citeproctext{#2}\cite{#1}\endgroup}
\makeatletter
 % allow citations to break across lines
 \let\@cite@ofmt\@firstofone
 % avoid brackets around text for \cite:
 \def\@biblabel#1{}
 \def\@cite#1#2{{#1\if@tempswa , #2\fi}}
\makeatother
\newlength{\cslhangindent}
\setlength{\cslhangindent}{1.5em}
\newlength{\csllabelwidth}
\setlength{\csllabelwidth}{3em}
\newenvironment{CSLReferences}[2] % #1 hanging-indent, #2 entry-spacing
 {\begin{list}{}{%
  \setlength{\itemindent}{0pt}
  \setlength{\leftmargin}{0pt}
  \setlength{\parsep}{0pt}
  % turn on hanging indent if param 1 is 1
  \ifodd #1
   \setlength{\leftmargin}{\cslhangindent}
   \setlength{\itemindent}{-1\cslhangindent}
  \fi
  % set entry spacing
  \setlength{\itemsep}{#2\baselineskip}}}
 {\end{list}}
\usepackage{calc}
\newcommand{\CSLBlock}[1]{\hfill\break\parbox[t]{\linewidth}{\strut\ignorespaces#1\strut}}
\newcommand{\CSLLeftMargin}[1]{\parbox[t]{\csllabelwidth}{\strut#1\strut}}
\newcommand{\CSLRightInline}[1]{\parbox[t]{\linewidth - \csllabelwidth}{\strut#1\strut}}
\newcommand{\CSLIndent}[1]{\hspace{\cslhangindent}#1}

\KOMAoption{captions}{tableheading}
\makeatletter
\@ifpackageloaded{caption}{}{\usepackage{caption}}
\AtBeginDocument{%
\ifdefined\contentsname
  \renewcommand*\contentsname{Table of contents}
\else
  \newcommand\contentsname{Table of contents}
\fi
\ifdefined\listfigurename
  \renewcommand*\listfigurename{List of Figures}
\else
  \newcommand\listfigurename{List of Figures}
\fi
\ifdefined\listtablename
  \renewcommand*\listtablename{List of Tables}
\else
  \newcommand\listtablename{List of Tables}
\fi
\ifdefined\figurename
  \renewcommand*\figurename{Figure}
\else
  \newcommand\figurename{Figure}
\fi
\ifdefined\tablename
  \renewcommand*\tablename{Table}
\else
  \newcommand\tablename{Table}
\fi
}
\@ifpackageloaded{float}{}{\usepackage{float}}
\floatstyle{ruled}
\@ifundefined{c@chapter}{\newfloat{codelisting}{h}{lop}}{\newfloat{codelisting}{h}{lop}[chapter]}
\floatname{codelisting}{Listing}
\newcommand*\listoflistings{\listof{codelisting}{List of Listings}}
\makeatother
\makeatletter
\makeatother
\makeatletter
\@ifpackageloaded{caption}{}{\usepackage{caption}}
\@ifpackageloaded{subcaption}{}{\usepackage{subcaption}}
\makeatother
\ifLuaTeX
  \usepackage{selnolig}  % disable illegal ligatures
\fi
\usepackage{bookmark}

\IfFileExists{xurl.sty}{\usepackage{xurl}}{} % add URL line breaks if available
\urlstyle{same} % disable monospaced font for URLs
\hypersetup{
  pdftitle={The World Bank, Women, Business and the Law},
  pdfauthor={Sehar Bajwa; Luca Carnegie},
  colorlinks=true,
  linkcolor={blue},
  filecolor={Maroon},
  citecolor={Blue},
  urlcolor={Blue},
  pdfcreator={LaTeX via pandoc}}

\title{The World Bank, Women, Business and the Law\thanks{Code and data
are available at: https://github.com/lcarnegie/replicationpaper.}}
\usepackage{etoolbox}
\makeatletter
\providecommand{\subtitle}[1]{% add subtitle to \maketitle
  \apptocmd{\@title}{\par {\large #1 \par}}{}{}
}
\makeatother
\subtitle{The Need for better checking of Data}
\author{Sehar Bajwa \and Luca Carnegie}
\date{February 20, 2024}

\begin{document}
\maketitle
\begin{abstract}
First sentence. Second sentence. Third sentence. Fourth sentence.
\end{abstract}

\section{Introduction}\label{introduction}

Despite the many decades' worth of progress in gender equality
throughout the world, one of the enduring types of gender discrimination
worldwide is in the eyes of the law. Though much progress has been made
in western, high-income countries, there remains vast swaths of the
world where women may not vote, work, or participate in the same
activities as men do, due to the presence of gendered laws. To trace the
progression of economic opportunities of women in the world's 190
countries, the World Bank constructed their ``Women, Business, and the
Law'' (WBL) dataset, which Hyland et. al make of primary interest in
their paper ``Gendered Laws and Women in the Workforce''. In Sections 2
and 3, we analyze and replicate some of the key graphs and insights from
their paper. Building on their work in Sections 4 and 5, we call into
question the data collection methods of the World Bank, finding data
from Statistics Canada that conflict with the data {[}or whatever Sehar
finds{]}, outlining the importance that accurate data have in being able
to fully understand the complex picture of women and the law throughout
the world.

\section{Data}\label{sec-data}

Hyland et al.'s original paper primarily makes use of the `Women,
Business, and the Law' dataset (WBL) from the World Bank, along with
some others to help compare and contextualize the data. In our
replication, however, we will be focusing on aspects of this dataset
only.

\subsection{Source}\label{subsec1-data}

The Women, Business, and the Law dataset is organized across thirty-five
aspects of the law, which are scored across eight indicators of four or
five binary questions. Each indicator represents a different phase of a
woman's career: Mobility, Workplace, Pay, Marriage, Parenthood,
Entrepreneurship, Assets, and Pension. Answers to the binary questions
were sourced, pro-bono, from respondents with expertise in laws on
family, labor and violence against women. It not clear where in their
countries' legal systems the respondents were when they were polled. The
indicators are as follows:

\begin{verbatim}
Mobility: Examines constraints on women’s freedom of movement. 
Workplace: Analyzes laws affecting a woman’s decision to work
Pay: Measures laws and regulations affecting a woman’s pay
Marriage: Assesses constraints related to marriage
Parenthood: Examines laws that affect women’s work after having children
Entrepreneurship: Assesses constraints to women starting and running a business
Assets: covers property ownership rights, inheritance rights (both for children and surviving spouses), authority of assets during marriage, and valuation of nonmonetary contributions. 
Pension: captures the equalization of retirement ages (with full and partial benefits as well as the mandatory retirement age) and whether periods of absence from employment due to childcare are accounted for in pension benefits. 

Indicator-level scores are obtained by calculating the unweighted average of the questions within that indicator and scaling the result to 100. Overall scores are then calculated by taking the average of each indicator, with 100 representing the highest possible score. Each binary question either contributes 20 or 25 points, depending on the indicator. The indicators and index mainly capture the effect labour laws have on the employment outcomes of women in the labour force. It does not take into account laws affecting their acquisition of human capital (e.g. required education laws) nor laws affecting reproductive rights, for instance, which Hyland et al. point to as an issue but concede would be difficult to quantify easily. The World Bank's methodology was designed as an easily replicable measure of the legal environment for women as entrepreneurs and employees, which, for a dataset of it's size and breadth, is comprehensive. In their paper, they then use the dataset to present 'stylized facts' which either unearth new facts about gendered laws or are quantifications of what they write were previously known qualitatively. 


Figures 1A, 1B, and 2 were replicated from the original paper. 


Figure 1A shows the average unweighted aggregate WBL index score in 2019, compared across regions. It shows notable differences between higher-income countries and the rest of the world, most notably between higher income countries and countries in the Middle East and North Africa - there is a almost 25 index-point difference between them. In the context of this dataset, it indicates that women in high-income countries enjoy about 36 percent more legal rights related to employment than those in the Middle East and North Africa. 

Figure 1B breaks down aggregate index for 2019 into it's constituent indicators and shows the global average score for each indicator. As Hyland et al pointed out, the 'Parenthood' Indicator was the lowest overall, meaning that globally, women generally enjoy fewer legal rights in employment after they have had children than before. 
\end{verbatim}

Figure 2 shows the progression of the aggregate WBL index over time,
broken down by region. As expected, high-income countries dominate in
index score, but interestingly Latin American women enjoyed more rights
than even European/Central Asian women between 1995-2000. Employment
rights for women in Sub-Saharan Africa have also seen the most growth
since tracking began in 1971. \#\# Methodology \{\#subsec2-data\}

\section{Results}\label{sec-results}

\section{Discussion}\label{discussion}

\subsection{First discussion point}\label{sec-point1}

\subsection{Second discussion point}\label{sec-point2}

\subsection{Third discussion point}\label{sec-point3}

\subsection{Weaknesses and next steps}\label{weaknesses-and-next-steps}

\newpage

\section{References}\label{references}

\phantomsection\label{refs}
\begin{CSLReferences}{0}{1}
\end{CSLReferences}

\newpage

\appendix

\section*{Appendix}\label{appendix}
\addcontentsline{toc}{section}{Appendix}



\end{document}
